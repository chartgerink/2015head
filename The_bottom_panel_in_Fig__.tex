The bottom panel in Fig 1 indicates there is a reporting tendency at the second decimal for p-values larger than $.01$. If no reporting tendencies existed, the distribution would show a reasonably smooth distribution. However, the depicted distribution violates this, where p-value frequencies drastically increase at each second decimal place in the distribution. A post-hoc explanation for this is that three decimal reporting of p-values has only been prescribed since 2010 in psychology \cite{American_Psychological_Association2010-qe}, where it previously prescribed two decimal reporting \cite{American_Psychological_Association1983-yf, American_Psychological_Association2001-uw}. Because reporting has occurred at the second decimal place for a long time and can be seen to have a substantial effect on the distribution, I think it is important to take this into account in the bin selection.

Head et al. selected the bins as indicated in the top panel in Fig 1, removing the second decimal. For their tests of p-hacking, they compared the bin frequency of the adjacent bins $.04<p<.045$ versus $.045<p<.05$. The original authors “suspect that many authors do not regard $p=.05$ as significant” \cite{Head_2015}, which is why they eliminate the second decimal from their analyses by using the selection criterion $<.05$. Previous investigation of p-values reported as exactly .05 revealed that 94.3\% of 236 cases interpret this as statistically significant \cite{Nuijten2015}. 

This contradicts the premise that most researchers do not interpret $p=.05$ as significant, which removes the reason for eliminating the second decimal. This is why I argue that the selection criterion should be $p\leq.05$ and not $p<.05$ and should look only at bins just below the second decimal. More specifically, because of reporting tendencies the analyses need to compare the frequencies below .04 and .05 (e.g., $.03875<p<.04$ versus $.04875<p<.05$ for binwidth .00125). This corresponds to the two bins shown in the bottom panel of Fig 1 at .04 and .05.
