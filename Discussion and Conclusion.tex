\section*{Discussion} 
A limitation of selecting the bins just below .04 and .05 is that these bins are non-adjacent and therefore might be less sensitive to detecting left-skew p-hacking. In light of this limitation I ran the original analysis but included the second decimal, which resulted in the comparison of $.04\leq p<.045$ versus $.045<p\leq.05$. This analysis also yielded no evidence for left-skew p-hacking, $Prop.=.457,p>.999,BF_{10}<.001$. 

A final limitation regards the selection of only exactly reported p-values. This might have distorted the p-value distribution due to minor rounding biases. Previous research has indicated that p-values are somewhat more likely to be rounded to .05 rather than to .04 \cite{Krawczyk2015-uh}. Therefore, selecting only exactly reported p-values might cause an underrepresentation of .05 values, because p-values are more frequently rounded and reported as $<.05$ instead of exactly (e.g., $p=.046$). This limitation also applies to the original paper by Head et al. and is therefore a general limitation to analyzing p-value distributions.



\section*{Conclusion}
The conclusion that there is a lack of evidence for left-skew p-hacking remains in this reanalysis, even in light of the limitations of the current reanalysis. 
These reanalyses indicate that the evidence for p-hacking is either underestimated (sensitivity reanalysis) or artefactual  (full reanalysis). The problem of second-decimal reporting bias is, in my opinion, something that cannot be neglected and therefore I conclude that the data provided by Head and colleagues yields no substantial evidence for widespread p-hacking throughout the sciences. This is in line with a previous reanalysis that indicated that the paper that initiated the line of research into p-hacking \cite{Masicampo2012} did not provide sufficient evidence for p-hacking after \cite{Lakens2014}. In sum, this reanalysis yields no evidence for widespread p-hacking.
  
  