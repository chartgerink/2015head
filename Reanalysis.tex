\section*{Reanalysis}
Results of the reanalysis indicate that no evidence for left-skew p-hacking remains when we take into account a second-decimal reporting bias. Initial sensitivity analyses using the original analysis script strengthened original results after eliminating DOI selection and using $p\leq.05$ as selection criterion instead of $p<.05$. However, as explained above, this result is confounded due to not taking into account the second decimal. Reanalyses across all disciplines showed no evidence for left-skew p-hacking, $Prop.=.417,p>.999, BF_{10}<.001$ for the Results sections and $Prop.=.358,p>.999,BF_{10}<.001$ for the Abstract sections. These results are not dependent on binwidth .00125, as is seen in Table \ref{tab1} where results for alternate binwidths are shown.  Separated per discipline, no binomial test for left-skew p-hacking is statistically significant in either the Results- or Abstract sections (see S1 File). This indicates that the effect found originally by Head and colleagues does not hold when we take into account that reported p-values show reporting bias at the second decimal.

\begin{table}[htbp]
    \begin{tabular}{cccc}
              &       & Abstracts & Results \\
              \hline
      Binwidth = .00125 & ($.03875-.04$) & 4597  & 26047 \\
          & ($.04875-.05$) & 2565  & 18664 \\
          & $Prop.$ & 0.358 & 0.417 \\
          & $p$     & >.999 & >.999 \\
          & $BF_{10}$  & <.001 & <.001 \\
    Binwidth = .005 & ($.035-.04$) & 6641  & 38537 \\
          & ($.045-.05$) & 4485  & 30406 \\
          & $Prop.$ & 0.403 & 0.441 \\
          & $p$     & >.999 & >.999 \\
          & $BF_{10}$  & <.001 & <.001 \\
    Binwidth = .01 & ($.03-.04$) & 9885  & 58809 \\
          & ($.04-.05$) & 7250  & 47755 \\
          & $Prop.$ & 0.423 & 0.448 \\
          & $p$     & >.999 & >.999 \\
          & $BF_{10}$  & <.001 & <.001 \\

    \end{tabular}
    \caption{Table 1. Results of reanalysis across various binwidths (i.e., .00125, .005, .01).} 
\end{table}