\section*{Reanalysis}
In order to take into account the two-decimal reporting tendency, I propose adjusting the frequency bins used in testing for p-hacking. The application of the Caliper test, where frequencies of two bins are compared, is maintained. However, instead of selecting the two final bins, the bins adjacent to round second decimals are chosen. Additionally, the binwidth is adjusted from .005 to .00125 for more precision and comparability with previous research \cite{Masicampo2012, Leggett2013}. This results in comparing the frequencies for the bins (.04875-.05) and (.03875-.04) in a one-tailed binomial test ($H_0: P \leq .5$).

The results of the full reanalysis indicate that no evidence for p-hacking remains when we take into account a second-decimal reporting bias. Across all disciplines, the test for p-hacking yields $P=.417,p>.999$ for the Results sections and $P=.358,p>.999$ for the Abstract sections. Separated per discipline, no binomial test for p-hacking is statistically significant in either the Results- or Abstract sections (see S1 File). This indicates that the effect found originally by Head and colleagues does not hold when we take into account that reported p-values show reporting bias at the second decimal.
  
  