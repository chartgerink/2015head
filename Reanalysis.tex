\section*{Reanalysis}
Results of the reanalysis indicate that no evidence for left-skew p-hacking remains when we take into account a second-decimal reporting bias. Initial sensitivity analyses using the original analysis script strengthened original results after eliminating DOI selection and using $p\leq.05$ as selection criterion instead of $p<.05$. However, as explained above, this result is confounded. The reanalysis across all disciplines showed no evidence for left-skew p-hacking, $Prop.=.417,p>.999, BF_{10}<.001$ for the Results sections and $Prop.=.358,p>.999,BF_{10}<.001$ for the Abstract sections. These results are not dependent on binwidth .00125, as is seen in Table \ref{tab1} where results for alternate binwidths are shown.  Separated per discipline, no binomial test for left-skew p-hacking is statistically significant in either the Results- or Abstract sections (see S1 File). This indicates that the effect found originally by Head and colleagues does not hold when we take into account that reported p-values show reporting bias at the second decimal.

\begin{table}[htbp]
    \begin{tabular}{cccc}
              &       & Abstracts & Results \\
              \hline
    \multicolumn{1}{l}{\multirow{5}[0]{*}{Binwidth = .00125}} & (.03875-.04) & 4597  & 26047 \\
    \multicolumn{1}{l}{} & (.04875-.05) & 2565  & 18664 \\
    \multicolumn{1}{l}{} & $Prop.$ & 0.358 & 0.417 \\
    \multicolumn{1}{l}{} & $p$     & >.999 & >.999 \\
    \multicolumn{1}{l}{} & $BF_{10}$  & <.001 & <.001 \\
    \multicolumn{1}{l}{\multirow{5}[0]{*}{Binwidth = .005}} & (.035-.04) & 6641  & 38537 \\
    \multicolumn{1}{l}{} & (.045-.05) & 4485  & 30406 \\
    \multicolumn{1}{l}{} & $Prop.$ & 0.403 & 0.441 \\
    \multicolumn{1}{l}{} & $p$     & >.999 & >.999 \\
    \multicolumn{1}{l}{} & $BF_{10}$  & <.001 & <.001 \\
    \multicolumn{1}{l}{\multirow{5}[0]{*}{Binwidth = .01}} & (.03-.04) & 9885  & 58809 \\
    \multicolumn{1}{l}{} & (.04-.05) & 7250  & 47755 \\
    \multicolumn{1}{l}{} & $Prop.$ & 0.423 & 0.448 \\
    \multicolumn{1}{l}{} & $p$     & >.999 & >.999 \\
    \multicolumn{1}{l}{} & $BF_{10}$  & <.001 & <.001 \\
    \end{tabular}
    \caption{Table 1. Results of reanalysis across various binwidths (i.e., .00125, .005, .01).} 
\end{table}
  
A limitation of selecting the bins just below .04 and .05 is that these bins are non-adjacent and therefore might be less sensitive to detecting left-skew p-hacking. In light of this limitation I ran the original analysis but included the second decimal, which resulted in the comparison of $.04\leq p<.045$ versus $.045<p\leq.05$. This analysis also yielded no evidence for left-skew p-hacking, $Prop.=.457,p>.999,BF_{10}<.001$.

A final limitation regards the selection of only exactly reported p-values. This might have distorted the p-value distribution due to minor rounding biases. Previous research has indicated that p-values are somewhat more likely to be rounded to .05 rather than to .04 (Krawczyk, 2015). Selecting only exactly reported p-values might therefore cause an underrepresentation of .05 values.