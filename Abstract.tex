Statistical significance seeking (i.e., p-hacking) is a serious problem for the validity of research, especially if it occurs frequently. Head et al. provided evidence for widespread p-hacking throughout the sciences, which would indicate that the validity of science is in doubt. Previous substantive concerns about their selection of p-values indicated they were too liberal in selecting all reported p-values, which would result in including results that would not be interesting to have been p-hacked. Despite this liberal selection of p-values Head et al. found evidence for p-hacking, which raises the question why p-hacking was detected despite it being unlikely a priori. In this paper I reanalyze the original data and indicate Head et al. their results are an artefact of rounding in the reporting of p-values.