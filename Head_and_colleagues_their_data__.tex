Head and colleagues their data analytic strategy focuses on comparing frequencies in the last and penultimate bins from .05 at a binwidth of .005. Based on the tenet that p-hacking introduces a left skew p-distribution \cite{Simonsohn2014}, evidence for p-hacking is present if the last bin has a sufficiently higher frequency than the penultimate one in a binomial test. For the overall dataset as in the sensitivity reanalysis, this yields the bins as in Figure 1. Note that the evidence from the sensitivity analyses were supposedly stronger than in the original paper, but this is hardly the case here.  