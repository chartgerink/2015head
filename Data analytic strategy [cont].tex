The bottom panel in Fig 1 indicates there is a reporting tendency at the second decimal for p-values larger than or equal to $.01$. If no reporting tendencies existed, the distribution would show a reasonably smooth distribution, resembling the distribution between $0$ and $.01$. However, the depicted distribution violates this, where p-value frequencies drastically increase at each second decimal place in the distribution. A post-hoc explanation for this is that three decimal reporting of p-values has only been prescribed since 2010 in psychology \cite{American_Psychological_Association2010-qe}, where it previously prescribed two decimal reporting \cite{American_Psychological_Association1983-yf, American_Psychological_Association2001-uw}. Because reporting has occurred to the second decimal place for a long time and can be seen to have a substantial effect on the distribution, I think it is important to take this into account in the bin selection.

Head et al. selected the bins as indicated in the top panel in Fig 1, removing the second decimal. For their tests of p-hacking, they compared the bin frequency of the adjacent bins $.04<p<.045$ versus $.045<p<.05$. The original authors “suspect that many authors do not regard $p=.05$ as significant” \cite{Head_2015}, which is why they eliminate the second decimal from their analyses by using the selection criterion $<.05$. Previous investigation of p-values reported as exactly .05 revealed that 94.3\% of 236 cases interpret this as statistically significant \cite{Nuijten2015}.

This contradicts the premise that most researchers do not interpret $p=.05$ as significant, which removes the reason for eliminating the second decimal. Consequently, only exactly reported p-values smaller than or equal to .05 were retained for the analyses, whereas Head et al. retained only exactly reported p-values smaller than .05. Moreover, because of reporting tendencies and the inclusion of the second decimal, the analyses need to compare the frequencies below .04 and .05 (e.g., $.03875<p<.04$ versus $.04875<p<.05$ for binwidth .00125). This corresponds to the two bins shown in the bottom panel of Fig 1 at .04 and .05.

In this paper, binomial proportion tests for left-skew p-hacking were conducted in both the frequentist and Bayesian framework, where $H_0:Prop.\leq.5$. The frequentist p-value gives the probability of the data if the null hypothesis is true, but does not quantify the probability of the null and alternative hypotheses. A Bayes Factor ($BF$) quantifies these latter probabilities, either as $BF_{10}$, the alternative hypothesis versus the null hypothesis, or vice versa, $BF_{01}$. A $BF$ of 1 indicates that both hypotheses are equally probable, given the data. In this specific instance, $BF_{10}$ is computed and values $>1$ can be interpreted, for our purposes, as: the data are more likely under p-hacking than under no p-hacking. $BF_{10}$ values $<1$ indicate that the data are more likely under no p-hacking than under p-hacking. The further removed from $1$, the more evidence in the direction of either one hypothesis, which were assumed to be equally likely in the prior distribution. For the current analyses, equal priors were assumed.