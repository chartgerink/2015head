In their original analyses, Head and colleagues use seven selection steps. First, they select only papers including NHST p-values. Second, only papers with a Digital Object Identifier (DOI) are retained. Third, only papers with p-values in the results section are retained. Fourth, only papers with non-zero authors are retained. Fifth, supplemental materials are removed. Sixth, only p-values smaller than .05 are retained (i.e., $<.05$). Finally, only exactly reported p-values are retained (i.e., $p=...$). Below, I evaluate their choices.

First, they select only papers including NHST p-values. 

Second, only papers with a DOI are retained, which results in the elimination of $84409$ p-values. However, there seems no substantive reason to eliminate papers without a DOI. In fact, this only results in eliminating p-values from older articles. DOIs were initiated in 1999 \cite{crossref2009} and have been issued for older publications, but most likely not at the same coverage as post-1999 publications. In other words, such a selection 

Third, only papers with p-values in the results section are retained. 

Fourth, only papers with non-zero authors are retained. 

Fifth, supplemental materials are removed. Sixth, only p-values smaller than .05 are retained (i.e., $<.05$). 

Finally, only exactly reported p-values are retained (i.e., $p=...$). 