Megan Head and colleagues provide a large collection of p-values that, from their perspective, indicates widespread statistical significance seeking (i.e., p-hacking) throughout the sciences. The analyses that form the basis of their conclusions operate on the tenet that p-hacked papers show p-value distributions that are left skew under .05 \cite{Simonsohn2014}. In this paper I evaluate their data analytic choices and strategy. I applaud their transparency in sharing both the data and analysis scripts, allowing for alternative data analytic perspectives.

In line with their original openness, I version controlled all my research efforts. Version control provides a timestamped history of the changes made to files, such as the analysis code or writing \cite{Ram2013}. This is comparable to track changes, but applied to computer files. Such version control is valuable for reproducing the research process and publicly available for this paper at \href{https://github.com/chartgerink/2015head}{Github}.

This paper is structured into four parts: (i) evaluation of data analytic choices made by Head and colleagues, (ii) reevaluating the results based on my alternative choices but the same data analytic approach (i.e., sensitivity reanalysis), (iii) evaluating the data analytic strategy, and (iv) reevaluating the results of a different data analytic strategy (i.e., strong reanalysis). 