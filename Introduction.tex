Megan Head and colleagues provide a large collection of p-values that, from their perspective, indicates widespread statistical significance seeking throughout the sciences. The analyses that form the basis of their conclusions operate on the tenet that p-hacked papers show p-value distributions that are left skew under .05 \cite{Simonsohn2014}. In this paper I evaluate their analytic approach and how this left skew is a sufficient, but not necessary condition for detecting p-hacking. I applaud their transparency in sharing both the data and analysis scripts, allowing for alternative data analytic perspectives.

In line with their original openness, I version controlled all my research efforts. Version control provides a timestamped history of the changes made to files, such as the analysis code or writing \cite{Ram2013}. This sort of track changes for files is valuable for reproducing the research process. The version control of this paper is freely accessible at \href{https://github.com/chartgerink/2015head}{Github}.

This paper is structured into three parts: (i) evaluation of data analytic choices made by Head and colleagues, (ii) reevaluating the results based on alternative choices, and (iii) evaluating the results of an alternative data analytic approach.