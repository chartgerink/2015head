Head and colleagues their data analytic strategy focuses on comparing frequencies in the last and penultimate bins from .05 at a binwidth of .005. Based on the tenet that p-hacking introduces a left skew p-distribution \cite{Simonsohn2014}, evidence for p-hacking is present if the last bin has a sufficiently higher frequency than the penultimate one in a binomial test. 

These analyses are run on exactly reported p-values, which are subject to reporting tendencies. Head and colleagues already eliminate one pernicious reporting tendency: inexact reporting. If included, these would artificially inflate the frequencies of reported p-values at the significance thresholds (e.g., $p<.05$ would be considered in the .05 frequency bin). Upon inspecting the frequency distribution of p-values below .05 in Fig. 1 (including .05), we clearly see that exactly reported p-values suffer from a reporting tendency at the second decimal place. It is common practice in science to report numeric values to the second decimal and p-values have not been the exception for long. Only in its most recent edition, did the American Psychological Association prescribe three decimal reporting of p-values \cite{AmericanPsychologicalAssociation2010}. Considering this only readily applies to the field of psychology, it seems plausible that the two-decimal reporting tendency seeps through in the current data. 

In order to take into account the two-decimal reporting tendency, I propose a minor adjustment to the data analytic strategy. I maintain the analytic approach, where frequencies of two bins e