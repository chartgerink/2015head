\section*{Data analytic strategy}
Head and colleagues their data analytic strategy focused on comparing frequencies in the last and penultimate bins from .05 at a binwidth of .005. Based on the tenet that p-hacking introduces a left skew p-distribution \cite{Simonsohn2014}, evidence for p-hacking is present if the last bin has a sufficiently higher frequency than the penultimate one in a binomial test. Applying the binomial test to two frequency bins has previously been used in publication bias research and is typically called a Caliper test \cite{gerber2010, kuhberger2014}.

Of vital importance in the Caliper test is the careful selection of the frequency bins that are compared. When inspecting p-hacking, selecting the two final bins at .05 is logical, if the p-values are not subject to any biases. However, p-values are subject to reporting tendencies, of which one is already eliminated by excluding the inexactly reported p-values. Upon inspecting the frequency distribution of p-values below .05 in Fig. 1, it is readily seen that exactly reported p-values suffer from a reporting tendency at the second decimal place. Post-hoc explanations for this include that it is common practice in science to report numeric values to the second decimal and that reporting of p-values is no exception. In psychology, three decimal reporting of p-values has only been prescribed since 2010 \cite{AmericanPsychologicalAssociation2010}. It therefore seems plausible that a second-decimal reporting tendency creeps into the current data and needs to be taken into account when evaluating the evidence for p-hacking. Such second-decimal reporting bias can be corrected for by selecting bins adjacent to round second decimals, as is done in the strong reanalysis below.