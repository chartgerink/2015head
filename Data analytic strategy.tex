\section*{Data analytic strategy}
Head and colleagues their data analytic strategy focused on comparing frequencies in the last and penultimate bins from .05 at a binwidth of .005. Based on the tenet that p-hacking introduces a left skew p-distribution \cite{Simonsohn2014}, evidence for p-hacking is present if the last bin has a sufficiently higher frequency than the penultimate one in a binomial test. Applying the binomial test to two frequency bins has previously been used in publication bias research and is typically called a Caliper test \cite{gerber2010, kuhberger2014}.

The two panels in Fig 1 describe the selection of p-values in the original and current paper. The top panel shows the selection made by Head et al. (i.e., $.04<p< .045$ versus $.045<p<.05$), where the right bin shows a slightly higher frequency than the left bin. This is what Head et al. found as evidence for p-hacking. However, if we expand the range and look at the entire distribution, we see that this is an unrepresentative part of the distribution of significant p-values.
  