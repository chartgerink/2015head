Head and colleagues their data analytic strategy focuses on comparing frequencies in the last and penultimate bins from .05 at a binwidth of .005. Based on the tenet that p-hacking introduces a left skew p-distribution \cite{Simonsohn2014}, evidence for p-hacking is present if the last bin has a sufficiently higher frequency than the penultimate one in a binomial test. 

These analyses are run on exactly reported p-values, which are subject to reporting tendencies. Head and colleagues already eliminate one pernicious reporting tendency: inexact reporting. If included, these would artificially inflate the frequencies of reported p-values at the significance thresholds (e.g., $p<.05$ would be considered in the .05 frequency bin). Upon  